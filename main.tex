\documentclass{standalone}

\usepackage{ctex}

\usepackage{tikz}
%% =============自定义绘制雷达蛛网图环境================
%% 参数说明:
%% 第1个参数是可选参数,表示每一维的最大能力值,默认为10
%% 第2个参数是必选参数,表示最大维度值
%% 第2个参数是必选参数,表示能力轴标签(不同标签间用逗号分割,注意与维
%% 度值一致)
% 在{radarweb}环境中进行雷达蛛网图绘制
  % 该环境定义的每维度上节点命名规则为:
  %
  % D<d>-<v>,
  %
  % 其中,<d>维度,取1到最大维度\RD之间的值
  % <v>是该维度的能力值,取0到\SU之间的值
  %
  % 例如,采用75%透明方式绘制(重叠部分将以混合颜色绘制)如下各维度能力值  
  % D1(认知): 9/10; D2(理解): 10/10; D3(应用): 10/10;
  % D4(分析): 9/10; D5(评价): 8/10; D6(创造): 10/10;
  % 的雷达蛛网图的直接绘制示例代码为:
  % \draw [color=red, line width=1.5pt, opacity=0.75] (D1-9) --
  % (D2-10) -- (D3-10) -- (D4-9) -- (D5-8) -- (D6-10) -- cycle;
  % 也可以使用循环进行绘制,示例代码为:
  % \draw[color=red, line width=1.5pt, opacity=0.75] (D1-9)
  % \foreach \X/\Y in {2/10, 3/10, 4/9, 5/8, 6/10}
  % {
  %   -- (D\X-\Y) 
  % }
  %   -- cycle;
%% 该代码来源于:http://www.texample.net/tikz/examples/spiderweb-diagram/
\newenvironment{radarchart}[3][10] %code={\doublespacing},
{\begin{tikzpicture}
    % {veclen(2,2)}
    \newcommand{\RD}{#2} % 最大维度
    \newcommand{\DU}{#1} % 最大能力值

    \newdimen\R % 雷达图半径
    \R=2.8cm
    \newdimen\L % 标签半径
    \L=3.3cm

    \newcommand{\A}{360/\RD} % 不同维度轴间的角度

    \path (0:0cm) coordinate (O); % 坐标原点

    % 绘制雷达蛛网
    \foreach \X in {1,...,\RD}
    {
      \draw (\X*\A:0) -- (\X*\A:\R);
    }    
    
    \foreach \Y in {0,...,\DU}
    {
      \foreach \X in {1,...,\RD}
      {
        % 从0度开始逆时针布置各轴
        \pgfmathsetlengthmacro\Z{\X - 1}
        %\pgfmathtruncatemacro{\Z}{\X - 1};
        \path(\Z*\A:\Y*\R/\DU) coordinate (D\X-\Y);
        \fill (D\X-\Y) circle (0.5pt);
      }
      
      \draw[opacity=0.1] (0:\Y*\R/\DU)
      \foreach \X in {1,...,\RD}
      {        
        -- (\X*\A:\Y*\R/\DU)
      }
      -- cycle;      
    }

    \foreach \Y in {0, 2, ..., \DU}
    {
      \pgfmathsetmacro\Z{\Y/\DU} % 坐标值(范围:[0-1])
      \node[yshift=0.12cm] (nummer) at (D1-\Y) {\tiny{\pgfmathprintnumber[fixed, precision=2]{\Z}}};
    }

    \draw[color=yellow!70!black, line width=1.0pt, opacity=0.8, dashed] (O) circle (0.6 * \R);
    
    % 绘制能力轴标签
    \foreach[count=\i] \dim in {#3}
    {
      % 从0度开始逆时针布置各轴标签
      \pgfmathsetlengthmacro\Z{\i - 1}
      %\pgfmathsetmacro\Z{\i - 1}
      \path (\Z*\A:\L) node (L\i) {\tiny{\dim}};
    }
  }%
{\end{tikzpicture}}
%% ==================================================


\begin{document}

\begin{radarchart}{6}{生存, 控制, 机动, 爆发, 防御, 攻击}% 
  % 路径采用75%透明方式绘制,因此重叠部分将以混合颜色绘制
  %
  % 示例1(红色)(能力值取XX/10)
  % D1(生存): 9/10; D2(控制): 4/10; D3(机动): 6/10;
  % D4(爆发): 7/10; D5(防御): 7/10; D6(攻击): 8/10;
  \draw[color=red, line width=1.5pt, opacity=0.5] (D1-9)
  \foreach \X/\Y in {2/8, 3/6, 4/7, 5/7, 6/8}
  {
    -- (D\X-\Y) 
  }
  -- cycle;

  % 也可以使用如下代码直接绘制(能力值取XX/10)
  % \draw [color=red,line width=1.5pt,opacity=0.75] (D1-9) -- (D2-8)
  % -- (D3-6) -- (D4-7) -- (D5-7) -- (D6-8) -- cycle;

  % 示例2(绿色)(能力值取XX/10)
  % D1(生存): 7/10; D2(控制): 5/10; D3(机动): 8/10;
  % D4(爆发): 7/10; D5(防御): 9/10; D6(攻击): 8/10;
  \draw[color=green, line width=1.5pt, opacity=0.5] (D1-7)
  \foreach \X/\Y in {2/5, 3/9, 4/8, 5/8, 6/8}
  {
    -- (D\X-\Y) 
  }
  -- cycle;

  % 示例3(蓝色)(能力值取XX/10)
  % D1(生存): 9/10; D2(控制): 8/10; D3(机动): 8/10;
  % D4(爆发): 7/10; D5(防御): 6/10; D6(攻击): 3/10;
  \draw[color=blue, line width=1.5pt, opacity=0.5] (D1-5)
  \foreach \X/\Y in {2/6, 3/8, 4/7, 5/9, 6/8}
  {
    -- (D\X-\Y) 
  }
  -- cycle;
\end{radarchart}
\end{document}

%%% Local Variables:
%%% mode: latex
%%% TeX-master: t
%%% End:
